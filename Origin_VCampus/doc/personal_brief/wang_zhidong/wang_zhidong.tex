\documentclass{article}
\usepackage{ctex}
\usepackage{makecell}
\usepackage{geometry}
\usepackage{multirow}
\usepackage{multicol}
\usepackage{fancyhdr}
\usepackage{longtable}
\usepackage{color}

\geometry{a4paper,left=25mm,right=20mm,top=25mm,bottom=25mm}

\title{个人项目小结报告}
\author{09021134~王智东}
\date{\today}

\begin{document}

\maketitle

\section{组内工作}
\begin{enumerate}
    \item
          \texttt{Server}:

          % 学生学籍信息管理模块:\\
          学生学籍信息实体类\texttt{Student}的编写\\
          学生学籍信息Student与服务器端新增学生学籍信息,更新学生学籍信息、删除学生学籍信息,查找学生学籍信息的接口\texttt{StudentStatusController}的实现

    \item
          \texttt{Client}:


          学生学籍信息显示界面的编写以及美化\\
          商店界面新增商品界面的编写以及美化\\
          商店界面修改商品界面的编写以及美化\\
          教务管理界面显示学生各科成绩的界面的编写以及美化
\end{enumerate}

在项目开始的第一周,经过组内开会讨论确定项目设计思路后,一起编写了虚拟校园系统设计说明书,我负责的模块初步定为学生学籍信息管理模块的后端开发。在第二周中我完成了后端相关接口的开发之后和前端对接之后测试通过,和组长沟通后转到了前端,去编写一部分前端界面,第三周在编写前端代码,完成相关页面的设计和美化,在收尾工作中,我开始责商店模块的前后端接口对接。
\section{任务完成情况总结}

学生学籍信息服务器端实体类的编写和测试在第二周完成,第三周学生学籍信息模块前后端对接完毕,经过测试能够正常工作,同时在第三周完成了学生学籍信息显示界面、商店界面新增商品界面、商店界面修改商品界面、教务管理界面显示学生各科成绩的界面的编写。第四周完成了商店模板中,新增商品,购买商品,修改商品信息模块的前后端对接。在新增商品模块的编写时,因为和财务处对接时的失误,导致登录用户的职责出错,由组内成员进行了调整和重构。

\section{工作小结与体会}


因为之前还没有真正的去使用工业界主流解决方案去参与过一个完整的前后端页面项目的开发,在这次的开发任务中,我对于前后端的职责任务有了更加深入的认识与理解,我学会了使用\texttt{Compose Desktop}来编写前端页面,同时发现了\texttt{JAVA}基于注解进行编程所带来的极大便利,以及通过\texttt{Hibernate}来更快更高效的完成后端数据库的交互操作。在团队的不断磨合、配合过程中,我对于\texttt{git}这一代码版本工具的使用也更加得心应手。

这次集训中,在我编写代码时产生困惑时,十分感谢组内成员的答疑与解惑,我从他们身上学到了很多新知识。组内同学合作配合顺利、愉快,这是一次宝贵的经验。

\vfill
\noindent\textcolor{red}{教师点评}【优~~~~良~~~~中~~~~及格~~~~不及格】
\end{document}
