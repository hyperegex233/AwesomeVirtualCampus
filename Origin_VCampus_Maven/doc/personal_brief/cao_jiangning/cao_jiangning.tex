\documentclass{article}
\usepackage{ctex}
\usepackage{makecell}
\usepackage{geometry}
\usepackage{multirow}
\usepackage{multicol}
\usepackage{fancyhdr}
\usepackage{longtable}
\usepackage{color}
\geometry{a4paper,left=25mm,right=20mm,top=25mm,bottom=25mm}
\title{个人项目小结报告}
\author{09021215~曹江宁}
\date{\today}
\begin{document}
\maketitle
\section{组内工作}
组内分工为纵向分工,负责教务管理模块的相关工作。
\section{任务完成情况总结}
Course类:这是教务模块的基础,用于记录所有开设的课程信息。每门课程都包括课程名称、课程ID、任课老师等关键信息。此外,Course类还具有两个重要的静态方法:

courseToMap:将课程对象转化为包含字符串键值对的Map对象,方便在系统中进行数据传输和存储。
mapToCourse:将包含键值对的Map对象转化为Course类对象,其目的是在系统中对课程信息进行操作和管理。

SelectedClass类:用来记录学生选修的课程以及相关的成绩信息。学生的一卡通号被用作唯一标识符,以确保数据的准确性和一致性。

TeachingEvaluation类:这个类用来跟踪学生对老师的评教记录。它使用ClassUuid和学生一卡通号相匹配,以追踪特定课程的评教结果。

TeachingAffairsController类:这是信息管理系统的控制器,负责管理和协调各个功能模块之间的数据传输和操作。
以下是我完成的方法:

·addCourse:用于向系统中添加新的课程信息。

·deleteCourse:允许管理员删除不再开设的课程。

·selectClass:学生可以用这个方法选修课程。

·searchInfo:允许用户根据关键词搜索相关信息。

·searchClass:用于查找特定课程的详细信息。

·updateClass:管理员可以使用此方法更新课程信息。

·recordGrade:老师可以录入学生的成绩。

·searchGrade:学生可以查看他们的成绩。

\par TeachingAffairsClient类:这个类负责控制前端界面,使用户可以通过图形用户界面访问系统的功能。以下是我完成的方法:

·exportStudentList:允许管理员导出学生名单。

·addCourse:允许管理员添加新课程。

·searchCourse:学生可以搜索课程,以了解其详细信息。

·selectClass:学生可以使用此功能选修课程。

·searchInfo:根据课程名称查找课程信息。

·recordGrade:此功能用于录入学生的成绩。

·updateCourse:管理员可以更新课程信息。

·deleteCourse:允许管理员删除不再需要的课程记录。

\section{工作小结与体会}
需求分析与项目规划:一个项目的完成,前期的规划很重要,不仅对中期的构建起到指导作用,而且对最终的呈现效果起到决定性的作用,可以说,项目规划是一个项目的灵魂。
\par 使用git管理代码可以有效方便地查看同组的同学做出的修改以及管理代码版本更迭。

\vfill
\noindent\textcolor{red}{教师点评}【优~~~~良~~~~中~~~~及格~~~~不及格】

\end{document}