\documentclass{article}
\usepackage{ctex}
\usepackage{makecell}
\usepackage{geometry}
\usepackage{multirow}
\usepackage{multicol}
\usepackage{fancyhdr}
\usepackage{longtable}
\usepackage{color}

\geometry{a4paper,left=25mm,right=20mm,top=25mm,bottom=25mm}

\title{个人项目小结报告}
\author{09021218~马晓龙}
\date{\today}

\begin{document}

\maketitle

\section{组内工作}

在本次项目中的主要工作包括:

\begin{itemize}
    \item 完成了基于 \texttt{Netty} 的基础 Socket 通信框架的实现,并使用 \texttt{Gson} 作为传输序列化库
    \item 构建了基于 \texttt{Hibernate} 的 ORM 层,实现高效的、面向对象的数据访问和操作
    \item 编写和完善了各个模块的相关后端逻辑,实现了基于注解和反射的路由机制,并完成非侵入式的鉴权逻辑
    \item 编写和完善了部分客户端页面与对接部分逻辑,通过协程等方式避免了 UI 线程阻塞等问题
    \item 完成了项目的 CI/CD 配置文件的编写,实现自动化工作流
\end{itemize}

\section{任务完成情况总结}

总体任务完成情况良好,基础框架可满足上层功能的实现需求,为相关业务逻辑提供了足够的灵活性。编写的客户端界面、服务端逻辑可完成相应需求,能够为使用者提供良好的用户体验。

\section{工作小结与体会}

\begin{itemize}
    \item 团队协作、沟通和文档十分重要,否则即使已有基础框架,也难以规范地实现功能
    \item 对于协作提交的代码需要及时 Review,及时找出实现中存在的问题并加以改进,而不是等使用时出现问题了再处理
    \item 对 \texttt{Compose} 这类声明式、基于状态的 UI 框架有了更深入的了解,学习了如何以状态为中心构建应用程序
    \item 对 \texttt{Java} 语言本身有了更深入的了解,学习了相关的设计范式和编码规范
\end{itemize}

\vfill
\noindent\textcolor{red}{教师点评}【优~~~~良~~~~中~~~~及格~~~~不及格】
\end{document}
