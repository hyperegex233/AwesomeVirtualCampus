\documentclass{article}
\usepackage{ctex}
\usepackage{makecell}
\usepackage{geometry}
\usepackage{multirow}
\usepackage{multicol}
\usepackage{fancyhdr}
\usepackage{longtable}
\usepackage{color}

\geometry{a4paper,left=25mm,right=20mm,top=25mm,bottom=25mm}

\title{个人项目小结报告}
\author{09021130~张博雅}
\date{\today}

\begin{document}

\maketitle

\section{组内工作}

\subsection{商店模块设计}
在项目开始的第一周,我和组内的同学们在开会讨论确定项目设计思路后,一起编写了虚拟校园系统设计说明书,其中我负责的部分是商店模块的设计。\\

通过搜集资料,并总结组内同学提出的设计思路,从模块背景、需求分析和系统设计三方面,对商店模块的设计进行了具体说明。
\subsection{商店模块后端代码编写}
从第二周开始,我们组开始编写代码,我负责的是编写商店模块的后端代码。\\

编写任务主要分为三方面:商店所需的实体类StoreItem类和StoreTransaction类,客户端代码StoreClient,服务端代码StoreController。目标实现功能有:对于购物者,能够浏览商品、选择商品、购买商品;对于商店管理员,能够添加商品、删除商品、更新商品、查看商品销售情况。

\section{任务完成情况总结}
在第四周,我基本完成了上述后端所需实体类和方法的编写,对商店模块的需求和设计也有了更加清晰深刻的理解。但由于对前端的需求了解不够,编写商店前端代码的同学为我的代码作了补充。
\section{工作小结与体会}
此次专业技能实训课是我第一次参与软件系统的开发,对软件的设计流程并不了解,对设计过程所需的Java语言和数据库等工具也不够熟悉。但在这四周里,我学习并掌握了很多新的技能。

\paragraph{应用需求分析}
要结合生活实际,分析用户、管理员等身份对于一个软件的需求,尽量使设计更加全面、更加人性化。

\paragraph{服务端与客户端设计}
在此次实训中,我对客户端与服务端的工作方式有了更清晰的理解。客户层是应用程序的用户接口部分,担负着用户与应用之间的对话功能。它用于检查用户的输入数据,显示应用的输出数据。服务层又叫功能层,相当于应用的本体,它负责将具体的业务处理逻辑编入程序中。

\paragraph{数据库的连接}
在服务端方法中,通过Hibernate数据库会话连接到数据库,查询并获取所需要的信息。

在这次实训过程中,我和组内的同学们的合作非常顺利、愉快,我从他们身上学到了很多,这是一次非常宝贵的经历。

\vfill
\noindent\textcolor{red}{教师点评}【优~~~~良~~~~中~~~~及格~~~~不及格】
\end{document}
