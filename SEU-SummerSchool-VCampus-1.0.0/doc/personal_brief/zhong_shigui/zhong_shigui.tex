\documentclass{article}
\usepackage{ctex}
\usepackage{makecell}
\usepackage{geometry}
\usepackage{multirow}
\usepackage{multicol}
\usepackage{fancyhdr}
\usepackage{longtable}
\usepackage{color}

\geometry{a4paper,left=25mm,right=20mm,top=25mm,bottom=25mm}

\title{个人项目小结报告}
\author{09021202~钟世贵}
\date{\today}

\begin{document}

\maketitle

\section{组内工作}
\begin{itemize}
      \item 软件设计说明书图书馆模块部分的编写.

      \item 图书馆功能模块的设计与实现.
\end{itemize}


\section{任务完成情况总结}
\begin{itemize}
      \item 软件设计说明书

            完成了软件设计说明书中图书馆模块的编写,在其中给出了图书馆功能模块的规划设计与实现要求,并给出Server端和Client端的类分析与具体方法,提供了模块使用流程图与设计说明.

      \item 图书馆功能模块

            依据设计及需求完成了图书馆功能模块中主要实体类及其方法的编写,合作完成Server端LibraryBookController类的编写,在其中给出了能够响应Client端各种需求的方法,完成Client端LibraryClient的编写,在其中给出与前端对接的功能需求的接口. 完成部分前端页面的编写,完成部分与后端的对接逻辑.

      \item 功能测试与反馈逻辑

            依据使用感受与习惯添加了多处搜索框的按键事件,添加给出了多处功能反馈与提示,增加了多处异常处理逻辑.


\end{itemize}


\section{工作小结与体会}
\begin{itemize}
      \item 小结

            在本次项目的开发与设计中负责完成图书馆功能模块,在开发过程中完成了图书馆后端接口的编写与部分前端界面的绘制,基本实现 了初代软件设计说明书上规划的功能与界面.

      \item 体会

            关于软件开发,通过参与本次项目开发,我们了解了软件开发的大致流程,更深入了解了CS架构及其开发过程,以及数据库、应用服务器与应用程序客户端的交互对接模式;关于开发环境与开发工具,通过这次的项目我们认识到软件的设计开发不应拘泥于某种模式,应该本着满足用户需求、简单易行、高效的原则,在开发过程中应该合理选取开发环境,我们在此次项目的开发过程中选择了IntelliJ IDEA进行开发,选用异步和事件驱动的高性能网络应用框架Netty库来实现Socket,得到了更好的性能与可扩展性,使用更加现代化、更加高效的响应式UI工具包Jetback Compose来进行前端页面的设计与绘制,得到了更加美观与现代的UI界面,在此过程中学习与认识了Netty库各种API的调用,Netty的底层原理、设计框架,以及kotlin语言;关于小组合作,小组的通力合作是完成一个优秀的项目的前提,在此次项目的开发合作中,我们在组长的带领下进行了合理的分工,使用Git来进行项目版本管理与代码托管,实现了高效且合理的合作模式.

\end{itemize}

\vfill
\noindent\textcolor{red}{教师点评}【优~~~~良~~~~中~~~~及格~~~~不及格】

\end{document}
