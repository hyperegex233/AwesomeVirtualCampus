% !TeX root = ../introduction.tex
\ifx\maindoc\undefined
\errmessage{This file must be input from a main document!}
\fi

\section{引言}

\subsection{编写目的}
本文档旨在详细阐述“虚拟校园 (VCampus)”项目的整体设计方案,明确其系统架构、模块划分与核心技术细节。

该系统旨在构建一个多功能的虚拟校园平台,核心功能模块包括:用户管理(含注册、登录与授权)、学生学籍管理、在线选课系统、数字图书馆、校园商店、线上银行以及一个集成的LLM智能聊天界面。

为确保系统的高效、稳定与可扩展性,本次设计遵循了一系列现代软件工程原则。系统整体采用客户端/服务器(C/S)架构,并通过Socket通信实现数据交互 。在业务实现上,我们遵循严格的分层设计(Layered Architecture)与模型-视图-控制器(MVC)模式 ,以实现业务逻辑与用户界面的有效分离。此外,项目还将采用分布式开发思想,通过多线程处理并发用户请求 ,为项目构建一个结构清晰、易于协作和维护的开发基础。

本文档是指导项目开发全过程的文件。开发团队严格遵循本文档所定义的架构、接口与规范进行编码、测试与集成工作,以确保精确实现上述各模块功能的过程中高效协作。并且本文档为学生、老师、管理员等用户提供针对该项目详实的使用说明。

\subsection{背景}
软件系统名称: 虚拟校园 (VCampus) 系统 。
项目任务来源: 本项目是“暑期学校专业技能实训课程”设定的课程设计任务 ,旨在通过构建一个功能完善的软件系统,锻炼和检验开发团队在需求分析、架构设计、团队协作及项目管理方面的综合能力。
开发者: 姓名
目标用户: 本系统的主要用户群体为虚拟校园环境中的在校学生、教职员工以及负责后台维护的系统管理员。
运行环境: 系统基于客户端/服务器 (C/S) 架构 ,服务器端程序需部署在安装了兼容 JDK 17 版本 Java 运行环境的服务器上 ,并连接至 MySQL 数据库 。客户端程序为桌面应用程序,可在主流个人计算机操作系统(如 Windows, macOS, Linux)上跨平台运行。

\subsection{定义}
\paragraph{系统}指本文档所描述的虚拟校园系统软件。
\paragraph{用户}包括学生、教师和系统管理员。

\subsection{参考资料}
\subsubsection{项目内部文件}

《暑期学校专业技能实训课程安排》:本文档是项目的任务来源规定了项目的总体要求、功能模块、技术栈约束及考核标准。

《软件设计说明书DEMO》:为本次设计说明书的编写提供了内容结构和格式参考。

\subsubsection{国家及行业标准}

GB/T 8567-2006 计算机软件文档编制规范:本设计说明书的结构和内容参考了该国家标准,以确保文档的规范性。

Google Java Style Guide: 作为项目编码规范的主要参考,用于统一团队的代码风格、命名约定和格式化标准,以提升代码的可读性和可维护性。

Oracle Code Conventions for the Java(TM) Programming Language: 官方Java编程语言代码约定,作为编码规范的补充参考。

\subsubsection{技术与架构设计文献}

《Effective Java》 (3rd Edition) by Joshua Bloch: Java 语言的权威实践指南,为编写出更清晰、健壮、可复用的高质量Java代码提供了宝贵的建议。

《设计模式:可复用面向对象软件的基础》 (Design Patterns: Elements of Reusable Object-Oriented Software): 经典的 GoF 设计模式参考手册,为项目中如 MVC、单例等模式的应用提供了理论指导。

《代码整洁之道》 (Clean Code: A Handbook of Agile Software Craftsmanship): 软件工程领域的经典著作,指导开发者编写易于理解、易于维护的代码,其思想贯穿于本项目的编码实践中。

《架构整洁之道》 (Clean Architecture: A Craftsman's Guide to Software Structure and Design): 深入阐述了软件架构设计的核心原则,如组件、层次、边界和依赖关系等,是本项目进行分层设计和模块解耦的重要理论指导。
