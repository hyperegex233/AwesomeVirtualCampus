% !TeX root = ../introduction.tex
\ifx\maindoc\undefined
\errmessage{This file must be input from a main document!}
\fi

\section{程序系统的分析}

\subsection{可行性分析}

本节旨在对“虚拟校园 (VCampus)”项目的可行性进行综合评估,从技术实现、经济效益和实际操作三个维度,判断项目是否具备开发条件。

从技术可行性角度看,本项目是完全可实现的。项目要求采用客户端/服务器(C/S)架构,并通过Java Socket及多线程技术进行开发,这些都是成熟且拥有丰富文档支持的主流技术 。开发环境(Eclipse、兼容JDK 1.7的Java版本)和数据库(Access/MySQL)均为标准、易于获取的工具,不存在技术壁垒 。虽然团队成员主要以C++为开发语言,但具备扎实的计算机科学基础,将现有编程知识迁移至Java语言,并掌握Swing、Socket等API不存在根本性困难。因此,在技术层面,本项目风险可控。

在经济可行性方面,我们需将常规的“投资回报率”调整为“投入产出比”来评估。本项目的主要“投入”是开发团队在课程期间投入的时间和精力,而非大量的资金。所需的开发软件均为开源或免费版本,硬件则基于团队成员的个人计算机,几乎无直接的财务成本。而项目的“产出”则具有极高的价值:它不仅是完成“暑期学校专业技能实训课程”的必要条件 ,更是一次宝贵的、完整的软件工程实践。通过本项目,团队成员将获得在分层架构设计、网络通信、数据库交互以及团队协作方面的综合实战经验 ,这对未来的职业发展是重要的无形资产。因此,本项目具备显著的经济可行性。

最后,就操作可行性而言,VCampus系统旨在将校园中分散的线下业务(如学籍管理、选课、图书馆借阅等)整合到统一的线上平台,这符合教育信息化的大趋势 。对于目标用户(学生、教师、管理员),该系统能显著简化办事流程,提高信息获取效率 。系统所模拟的业务流程均源于真实的校园场景,易于用户理解和接受 。只要用户界面设计得足够直观易用,系统在理论上的校园环境中推广和使用,不存在明显的操作障碍。

综上所述,VCampus项目在技术、经济和操作层面均被证明是可行的。项目团队具备完成开发所需的技术学习和实践能力,项目成果能带来远超投入的知识回报,且其产品定位符合实际应用场景的需求。因此,项目具有可行性。

\subsection{需求分析}

本章节将详细定义VCampus系统的功能性需求与非功能性需求,作为后续系统设计、开发、测试及验收的基准。

\subsubsection{功能需求}

功能需求详细描述了系统应为用户提供的各项服务和功能。

1.用户管理模块(此模块是系统的基础,为所有用户提供身份认证与管理服务)
注册: 新用户(学生或管理员)能够通过界面输入必要信息(如用户名、密码、角色)创建新账户 。
登录/登出: 已注册用户可以通过唯一的用户名和密码登录系统,访问其权限对应的功能 。用户也可以随时安全退出登录状态 。
授权: 系统需根据用户角色(学生、管理员)区分权限,确保不同角色只能访问其被授权的功能模块 。
2. 学生学籍管理模块(该模块主要面向管理员,用于管理学生的基本信息)
信息录入与修改: 管理员可以添加、删除和修改学生的学籍信息。
信息查询: 支持通过学号、姓名等关键信息快速查询学生资料。
列表视图:通过列表查看所有学生的信息。
3. 选课系统模块(此模块为学生提供在线选课服务,是系统的核心业务功能之一)
课程信息浏览: 学生可以查看当前学期所有可选课程的列表,包括课程名称、授课教师、学分、学时等信息。
在线选课/退课: 在规定时间内,学生可以根据个人培养计划选择或退选课程。
个人课表查询: 学生可以查看自己已选课程的时间和地点安排。
4. 图书馆模块(模拟校园图书馆的核心功能)
图书查询: 用户可以根据书名、作者等信息检索馆藏图书。
借阅/归还: 模拟图书的借阅和归还流程。
图书信息浏览:可以通过列表视图浏览图书信息。
5. 校园商店与银行模块(提供虚拟的校园生活服务)
商店: 用户可以浏览商品并使用虚拟货币进行购买。
银行: 提供虚拟货币的存取款、转账等基本银行服务。
6. LLM聊天界面模块(集成一个基于大语言模型的智能问答界面,为用户提供校园信息咨询等服务)

\subsubsection{非功能需求}

非功能需求定义了系统的质量属性和在运行环境中必须遵守的约束。

性能需求: 系统应具备良好的响应能力。在高并发场景下(例如选课高峰期),服务器端需能通过多线程机制有效处理多个客户端的同时请求,确保系统不出现严重延迟或无响应。

兼容性需求: 系统开发必须严格遵守技术栈约束。后端逻辑需使用Java语言编写,JDK版本必须兼容1.7 。数据库可选用Access或MySQL,但需保证SQL语句的通用性 。

可靠性与健壮性: 客户端与服务器端的Socket通信必须建立完善的异常处理机制,能够处理网络中断、超时等常见问题,避免程序因网络波动而崩溃 。

安全性需求: 用户密码等敏感信息在数据库中存储时不应为明文。客户端与服务器之间传输的所有对象都必须实现可序列化接口,并保持版本一致性,以防止数据传输错误 。

代码规范与可维护性: 项目代码必须遵循统一的编码规范(如Google Java Style Guide),并为主要的类和方法提供标准的JavaDoc注释 。系统的设计应严格遵循分层原则,确保各层之间职责清晰、耦合度低,便于后续的功能扩展与维护 。

\subsection{开发设计环境}

\subsubsection{IntelliJ IDEA}

本项目计划采用IntelliJ IDEA作为核心集成开发环境(IDE)。该工具由JetBrains开发,广泛应用于Java项目,其突出优势体现在智能代码补全、动态错误检测、强大的重构功能以及内建的版本控制系统集成,在java编程方面具有较大的优势。尽管实训课程安排中提及了Eclipse ,但经过团队的权衡与考量,IntelliJ IDEA被认为是更优的选择。虽然Eclipse在处理多项目工作区和增量编译方面具有一定优势,但其插件生态相对不活跃,且在Java代码的实际编辑体验、重构辅助和智能代码生成方面,Intelli-J IDEA具有明显领先优势 。对于一个时间有限的暑期实训项目,快速、高效地编写业务逻辑代码至关重要。IntelliJ IDEA所提供的卓越编辑辅助功能能够显著加速开发流程,其与Maven的深度集成也确保了项目构建过程的稳定性和可重复性。

IntelliJ IDEA在此项目之中所体现的核心优势体现在以下几个方面:

1.智能编码辅助与动态分析: IntelliJ IDEA提供强大的智能代码补全、动态错误检测和即时代码质量分析功能。这些功能能够实时识别潜在的代码缺陷和不规范之处,并提供优化建议。这不仅加快了编码速度,也从源头保障了代码的健壮性和可维护性 。
2.强大的重构功能: 对于多人协作的“V-campus”项目而言,代码结构会随着开发进程不断演进。IntelliJ IDEA提供的强大、安全的自动化重构功能(如类更名、方法提取等),能够确保在修改项目结构时,关联引用能够被正确更新,从而有效维护了代码库的完整性 。
3.深度集成版本控制系统: 该IDE内置了对Git、Subversion等主流版本控制系统的深度支持。开发人员无需离开IDE环境即可完成代码提交、分支管理、冲突解决等版本控制操作,极大地简化了团队协作流程 。

此外,IntelliJ IDEA也对Maven和Gradle等构建工具提供了深度支持,能够有效管理项目依赖与构建流程,显著提升开发效率与代码可维护性。

\subsubsection{JDK 17}

本项目的核心开发语言环境选用Java Development Kit (JDK) 17。该版本由Oracle指定为长期支持(LTS)版本,这代表着它将获得长达数年的稳定安全、性能和错误修复更新。JDK 17在语言特性、运行性能及安全性方面具备多项改进,如引入了Sealed Classes、Pattern Matching for Switch等新语法特性,增强了类型系统的表达力与代码的简洁性。其优化的垃圾回收机制和提升的编译效率也为系统的高并发处理与稳定运行提供了坚实基础。同时该版本兼容性较强,也更加稳定。

选择LTS版本有助于为“虚拟校园”系统的长期稳定运行提供坚实保障,避免了因使用非LTS版本而可能面临的频繁版本升级和兼容性风险。 JDK 17在语言特性和平台性能方面也进行了多项重要改进。其中,对于本项目开发尤为有价值的特性包括:

1.密封类(Sealed Classes): 该特性允许开发者明确地声明一个类或接口可以被哪些特定的子类继承或实现。这增强了类型系统的表达能力,为系统中的实体类(如User及其子类Student、Teacher、Admin)提供了更严谨的设计约束,从而提升了代码的安全性和可维护性 。
2.switch表达式的模式匹配(Pattern Matching for switch): 这一预览特性简化了处理不同数据类型或状态的逻辑分支。通过将instanceof表达式和switch语句相结合,它能够显著减少代码冗余,提高代码的可读性与简洁性 。在处理“虚拟校园”系统中不同类型的消息(Message)或用户请求时,这一特性可以极大地简化业务逻辑层的开发 。

\subsubsection{MySQL 8.4.6 LTS}

数据库管理系统方面本软件设计采用MySQL 8.4.6 LTS版本。该版本不仅完全支持ACID事务,还提供诸如窗口函数、通用表表达式(CTE)和JSON字段增强功能等高级特性,极大丰富了数据操作与查询的能力。

此外,MySQL 8.4.6在身份验证、数据加密与权限管理等方面引入多项安全机制,保障了数据的机密性与完整性,非常适用于对数据一致性要求较高的校园管理系统。此版本不仅完全支持ACID事务,确保了在处理选课、学生学籍变更等核心业务中数据的一致性与完整性,还提供了注入公共表达式以及窗口函数等高级特性,极大丰富了数据操作能力。

\subsubsection{Java FX}

图形用户界面(GUI)部分基于Java FX框架实现。作为一个现代、跨平台的UI框架,Java FX提供了丰富的可视化组件、灵活的布局控制(FXML)以及CSS样式定制能力,能够构建响应迅速、视觉效果良好的桌面应用程序。其与JDK的紧密集成简化了部署过程,同时具备与底层图形硬件的良好协同,显著提升了客户端的用户体验。
其核心优势首先体现在现代化与跨平台,这代表了JavaFX提供了丰富的可视化组件和灵活的布局控制,能够构建响应迅速、视觉效果良好的桌面应用程序 。其跨平台特性确保了客户端系统在不同操作系统上(如Windows, macOS, Linux)表现出一致的用户体验。

其次体现在了设计与逻辑分离,JavaFX支持使用FXML(一种基于XML的声明式UI语言)来定义界面布局,并使用CSS样式表来定制界面风格 。这种模式将界面设计与后端业务逻辑进行了清晰的分离,完全契合了项目中强调的“分层设计”原则(MVC) 。它允许团队中的UI设计师与后端开发人员独立工作,从而显著提升开发效率。

\subsubsection{Maven}

本项目还计划采用Apache Maven作为项目构建与依赖管理工具。Maven是一款基于项目对象模型(POM)的软件项目管理工具,可自动化完成项目的编译、测试、打包及部署等生命周期管理任务。通过规范的目录结构和统一的依赖管理机制,Maven显著提升了项目的可维护性与团队协作效率。

其强大的插件生态系统和与主流IDE(如IntelliJ IDEA)的无缝集成,进一步确保了构建过程的可重复性和环境一致性,为大型多模块项目的依赖控制与构建流程提供了稳定基础。
