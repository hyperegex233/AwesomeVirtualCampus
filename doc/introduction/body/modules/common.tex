% !TeX root = ../../introduction.tex
\ifx\maindoc\undefined
\errmessage{This file must be input from a main document!}
\fi

\section{公共模块设计说明}
公共模块(Common Module),作为“V-Campus”系统整体架构的基石,其战略定位在于承载客户端与服务器端之间所有共享的核心组件、数据模型、抽象接口和工具类,其设计质量直接决定了系统的可维护性、可扩展性以及客户端与服务器端协作的一致性。在项目所遵循的经典分层架构(view、biz/srv、vo、dao)中,此模块主要扮演了vo(值对象)层的角色,并集成了跨层、跨应用的通用服务与辅助功能。其核心价值在于通过单一代码源实现模型与接口的同步,从而规避了分布式开发中常见的类定义不一致问题,确保了数据模型在网络传输中的完整性与互操作性。

这种中心化的设计不仅极大地提升了代码复用率,也为后续的团队协作、功能扩展和系统维护奠定了坚实的基础,体现了项目团队在架构设计层面的深思熟虑 。本系统严格遵循面向接口编程与关注点分离的原则,将那些在系统多处复用、尤其是需要在网络两端共享的核心数据结构、工具方法及基础服务抽象为独立的公共模块。这些模块被组织在独立的Java项目中,并通过Maven依赖管理机制为客户端和服务器端项目所引用,确保了二进制级别的一致性,彻底避免了因类定义版本差异而引发的序列化异常等潜在问题。所有在此模块中定义的、需要在网络中进行传输的对象,均严格实现了java.io.Serializable接口,这是保障基于Java对象流的Socket通信能够正常工作的先决条件。
\subsection{消息信息模型}
Message类,用于在客户端和服务器之间传递信息。它封装了消息类型、发送者、接收者、内容等核心数据,是整个C/S架构中通信协议的核心载体与抽象,它被设计为一个高度通用且可扩展的消息信封,旨在封装所有类型的客户端与服务器间的交互信息。

其重要性在于它统一了系统中诸如登录请求、课程数据查询、借阅指令等所有异构消息的传输格式,使得网络通信层能够以一致的方式处理各种业务请求与响应。该类的设计包含了多个关键字段,这里选取一些较为关键的进行展示如下所示:

uid作为全局唯一标识符(GUID),用于唯一标识每一条消息,这对于请求-响应的匹配、消息去重与异步通信下的日志追踪至关重要;

name字段定义了消息的可读名称,便于调试与日志记录;

type是一个枚举类型(例如MessageType.COMMAND, MessageType.DATA, MessageType.AUTH等),用于在接收端进行初步的消息路由,决定由哪个业务处理器(Handler)来接手后续处理;

而statusCode字段承载了操作结果的状态信息,借鉴HTTP状态码的理念(如200表示成功,404表示资源未找到,500表示服务器内部错误),使得客户端能标准化地解析处理结果;

data是一个声明为Object类型的成员,它承载了消息的实际业务载荷,其具体类型可由业务逻辑决定(如一个User对象、一个Course列表或一个简单的字符串),但其必须实现Serializable接口;

最后的sender字段记录了消息的发起方标识,通常为用户名或用户ID,用于服务端的身份验证与审计日志。通过对这些字段的组合与填充,Message类能够灵活适应从用户认证到数据同步等各类应用场景。各个关键字段所包含的特征可由下表直观展示:


\begin{tabular}{lllll}
    \toprule
    \textbf{序号} & \textbf{名称} & \textbf{类型}    & \textbf{约束}            & \textbf{备注} \\
    \midrule
    1           & uid           & Long           & 唯一性        &标识符\\
    2           & name          & String         & 6-16个字符    &名称  \\
    3           & type          & Enumberation   & 命令、数据     &类型 \\
    4           & statuscode    & String         &               &状态码           \\
    5           & data          & Object         & 可序列化       &传输数据   \\
    6           & sender        & Object/String  &               &发送者/用户名\\
    ...         & ...           & ...            &...            &...\\
    \bottomrule
\end{tabular}
\subsection{用户信息模型}
\subsection{工具类}
\subsubsection{数据库工具类}
\subsubsection{密码工具类}


% \subsection{Message}
% \subsection{User}
% \subsection{数据库工具类DbHelper}
% \subsection{数据存取类DAO}
% \subsection{工具类}
% \subsection{其他类}
