% !TeX root = ../../introduction.tex
\ifx\maindoc\undefined
\errmessage{This file must be input from a main document!}
\fi

\section{公共模块设计说明}
公共模块(Common Module),作为“V-Campus”系统整体架构的基石,其战略定位在于承载客户端与服务器端之间所有共享的核心组件、数据模型、抽象接口和工具类,其设计质量直接决定了系统的可维护性、可扩展性以及客户端与服务器端协作的一致性。在项目所遵循的经典分层架构(view、biz/srv、vo、dao)中,此模块主要扮演了vo(值对象)层的角色,并集成了跨层、跨应用的通用服务与辅助功能。其核心价值在于通过单一代码源实现模型与接口的同步,从而规避了分布式开发中常见的类定义不一致问题,确保了数据模型在网络传输中的完整性与互操作性。

这种中心化的设计不仅极大地提升了代码复用率,也为后续的团队协作、功能扩展和系统维护奠定了坚实的基础,体现了项目团队在架构设计层面的深思熟虑 。本系统严格遵循面向接口编程与关注点分离的原则,将那些在系统多处复用、尤其是需要在网络两端共享的核心数据结构、工具方法及基础服务抽象为独立的公共模块。这些模块被组织在独立的Java项目中,并通过Maven依赖管理机制为客户端和服务器端项目所引用,确保了二进制级别的一致性,彻底避免了因类定义版本差异而引发的序列化异常等潜在问题。所有在此模块中定义的、需要在网络中进行传输的对象,均严格实现了java.io.Serializable接口,这是保障基于Java对象流的Socket通信能够正常工作的先决条件。
\subsection{消息信息模型}
Message类,用于在客户端和服务器之间传递信息。它封装了消息类型、发送者、接收者、内容等核心数据,是整个C/S架构中通信协议的核心载体与抽象,它被设计为一个高度通用且可扩展的消息信封,旨在封装所有类型的客户端与服务器间的交互信息。

其重要性在于它统一了系统中诸如登录请求、课程数据查询、借阅指令等所有异构消息的传输格式,使得网络通信层能够以一致的方式处理各种业务请求与响应。该类的设计包含了多个关键字段,这里选取一些较为关键的进行展示如下所示:

uid作为全局唯一标识符(GUID),用于唯一标识每一条消息,这对于请求-响应的匹配、消息去重与异步通信下的日志追踪至关重要;

name字段定义了消息的可读名称,便于调试与日志记录;

type是一个枚举类型(例如MessageType.COMMAND, MessageType.DATA, MessageType.AUTH等),用于在接收端进行初步的消息路由,决定由哪个业务处理器(Handler)来接手后续处理;

而statusCode字段承载了操作结果的状态信息,借鉴HTTP状态码的理念(如200表示成功,404表示资源未找到,500表示服务器内部错误),使得客户端能标准化地解析处理结果;

data是一个声明为Object类型的成员,它承载了消息的实际业务载荷,其具体类型可由业务逻辑决定(如一个User对象、一个Course列表或一个简单的字符串),但其必须实现Serializable接口;

最后的sender字段记录了消息的发起方标识,通常为用户名或用户ID,用于服务端的身份验证与审计日志。通过对这些字段的组合与填充,Message类能够灵活适应从用户认证到数据同步等各类应用场景。各个关键字段所包含的特征可由下表直观展示:
\begin{table}[H]
        \centering
            \setlength{\tabcolsep}{6mm}{
                \begin{tabular}{ccccc}
                \hline
                序号    & 名称    & 类型    & 约束    & 备注\\
                \hline
                1 & \texttt{uid} & \texttt{long}   & 唯一性 & 标识符\\
                2 & \texttt{name} & \texttt{String} & 6-16个字符 & 名称\\
                3 & \texttt{type}  & \texttt{Enumberation} & 命令、数据 & 类型\\
                4 & \texttt{statuscode}  & \texttt{String} &  & 状态码\\
                5 & \texttt{data}   & \texttt{Object} & 可序列化 & 传输数据\\
                6 &\texttt{sender}  & \texttt{Object/String} &  & 发送者/用户者\\
                7 &...  & ... & ... & ...\\
                \hline
        \end{tabular}}
    \end{table}
\subsection{用户信息模型}
User类即为用户实体类,它包含了用户的ID、用户名、密码、昵称等属性,是系统核心业务实体的基础抽象,同时建模了虚拟校园环境中所有参与者的基本属性和行为。作为一个将在网络中传输并在持久化层存储的实体,它同样实现了Serializable接口。该类不仅包含了最基本的身份认证信息(如唯一标识用户的id和经过哈希加密处理的密码pwd),还包含了描述性属性如age。

尤为关键的是,它通过一个role字段(通常为枚举类型,如UserRole.STUDENT, UserRole.TEACHER, UserRole.ADMIN)来实现基于角色的访问控制(RBAC),该系统权限体系的基石。在后续的设计中,考虑到不同角色用户可能存在行为与属性的差异,可以采用继承机制对User类进行拓展,派生出Student、Teacher和Administrator等子类,从而通过多态特性更精确地建模现实世界。此外,该类应遵循JavaBean规范,提供所有属性的getter和setter方法,并重写toString(), equals()和hashCode()方法,以确保其能够方便地在各种业务逻辑、数据展示及集合操作中被正确处理。其所包含的一些关键字段如下表直观可视:
\begin{table}[H]
        \centering
            \setlength{\tabcolsep}{6mm}{
                \begin{tabular}{ccccc}
                \hline
                序号    & 名称    & 类型    & 约束    & 备注\\
                \hline
                1 & \texttt{id} & \texttt{String}   &  & 登录名\\
                2 & \texttt{pwd} & \texttt{String} & 6-16个字符 & 密码\\
                3 & \texttt{age}  & \texttt{Integer} & 非0 & 年龄\\
                4 & ...  & ... & ... & ...\\
                \hline
                \end{tabular}}
\end{table}
\subsection{工具类}
工具类提供一些静态方法,用于处理如日期格式化、字符串校验等通用功能。同时也集合了系统中广泛使用的、与特定业务无关的通用辅助方法。这些类通常被设计为最终类(final class)并包含私有构造方法,以防止被实例化或继承,其所有方法均为静态方法。

常见的工具类包括:StringUtils,提供字符串的非空判断、 trimming、格式化等操作;DateUtils,处理日期与字符串之间的转换、日期计算等;SecurityUtils,负责密码的加盐哈希(如使用BCrypt算法)、生成令牌等安全相关操作;LogUtils,提供统一的日志记录接口,包装了如Log4j或SLF4J等日志框架,确保整个系统日志输出格式和行为的一致性;JsonUtils,基于诸如Jackson或Gson库,提供Java对象与JSON字符串之间序列化与反序列化的能力,这对于未来可能扩展的RESTful接口或日志存储尤为重要。这些工具类的存在避免了通用代码的重复,保证了常用操作实现的一致性,是提升代码质量与开发效率的关键。
\subsubsection{数据库工具类DbHelper}
DbHelper类是一个专注于封装底层数据库连接与操作细节的基础设施类。其首要职责是管理数据库连接的生命周期,这是通过实现一个线程安全的数据库连接池来达成的,该池化机制能有效避免频繁创建和关闭连接所带来的巨大性能开销,从而提升系统在高并发场景下的吞吐能力。

该类通常被设计为单例模式,以保证整个应用程序内资源管理的统一性,它对外提供了一系列静态方法,如getConnection()用于从连接池中获取一个可用连接;executeQuery(String sql, Object... params)用于执行带参数的SQL查询语句,并返回ResultSet;executeUpdate(String sql, Object... params)用于执行插入、更新、删除等操作,返回受影响的行数;最后releaseConnection(Connection conn)并非真正关闭连接,而是将其返还给连接池以供复用。

此外,该类还承担了处理SQL异常、事务管理(如开启、提交、回滚事务)的职责,并将底层的SQLException转换为更易于上层业务逻辑处理的统一异常类型,从而将业务代码从繁琐的JDBC样板代码和资源管理中彻底解放出来。
\subsubsection{密码工具类}
\subsubsection{其他类}
除了上述所提到的项目的公共设计模块之中所计划涉及到的一些核心类,公共模块中还可能包含一系列支撑系统运行的其他重要类。下面就举出一些经典例子:Constants类集中定义了整个系统中使用的全局常量,如数据库连接字符串、服务器IP地址与端口号、各种状态码、文件路径等,这有利于维护和避免魔法数字的出现。AppException是一个自定义的、系统级的运行时异常基类,所有其他业务异常均可继承自此异常,这使得我们可以通过统一的异常处理机制(如AOP)来捕获和处理异常,并向用户返回友好的错误信息。ConfigLoader类专门负责从外部配置文件(如.properties或.xml文件)中加载配置信息,使系统具备良好的可配置性,无需修改代码即可适应不同的部署环境。这些类与前述模块共同协作,构成了一个健壮、灵活且易于维护的系统基础架构。

% \subsection{Message}
% \subsection{User}
% \subsection{数据库工具类DbHelper}
% \subsection{数据存取类DAO}
% \subsection{工具类}
% \subsection{其他类}
