% !TeX root = ../../introduction.tex
\ifx\maindoc\undefined
\errmessage{This file must be input from a main document!}
\fi

\section{公共模块设计说明}
\subsection{消息信息模型}
\subsubsection{请求结构Request}
我们计划自定义一个通用的请求格式,而且所有客户端发往服务器的请求都应当遵循这个结构,结构主要包括目标服务路径(uri)、携带的参数(params)以及用于身份验证的会话信息(session)。这样设计的好处是前后端交互格式统一,便于扩展和调试。
\subsubsection{响应结构Response}
在服务器处理完请求后,会返回一个结构固定的响应,包含操作状态(status)、提示信息(message)、实际返回的数据(data)以及更新后的会话信息(session)。这种结构有助于客户端快速判断请求结果并提取所需数据。
\subsection{用户信息模型}
系统中所有用户(包括学生、教师、管理员等)的基本信息都通过一个用户模型进行管理,涵盖账号、姓名、联系方式、权限角色等常用字段,为各功能模块提供基础数据支持。
\subsection{工具类}
工具类核心部分主要分为两块,即为密码工具类以及数据库工具类。
\subsubsection{密码工具类}
本工具类则负责用户密码的加密和验证。本项目计划采用  BCrypt 算法来对明文密码进行哈希处理,设置成本因子,生成随机盐值后使用EksBlowfish 算法计算生成最终的哈希字符串,同时在程序之中使用verify() 方法用于校验传入的密码是否与存储的哈希值匹配。
\subsubsection{数据库工具类DbHelper}
这个工具类主要封装了数据库连接及会话管理的基本操作,其提供了一个静态方法 init() 用于初始化数据库连接并返回 Hibernate 的 Session 对象,避免在各处重复编写连接代码。

% \subsection{Message}
% \subsection{User}
% \subsection{数据库工具类DbHelper}
% \subsection{数据存取类DAO}
% \subsection{工具类}
% \subsection{其他类}
